\documentclass[12pt,letter]{article}
\usepackage{mathptmx} % added for time new roman font
\usepackage[left=1in,right=1in,top=1in,bottom=1in]{geometry}
\usepackage[latin1]{inputenc}


\usepackage{amsmath}




% defines all example enviorment
\usepackage[framemethod=tikz]{mdframed} % added for the box around examples
\newtheorem{ex}{Example}
\numberwithin{ex}{section} % allows for the use of example numbers that lign up with the section numbers
\newenvironment{example}{\begin{mdframed}[middlelinewidth=0.5mm]\begin{ex}\normalfont}{\end{ex}\end{mdframed}}

% defines the quotation enviorment 
\usepackage{xcolor}
\newcommand{\quotebox}[2]{\begin{center}\fcolorbox{white}{blue!15!gray!15}{\begin{minipage}{0.9\linewidth}\vspace{10pt}\center\begin{minipage}{0.8\linewidth}{\space\Huge``}{#1}{\Huge''}{\break\null\hfill} {\small #2}  \end{minipage}\medbreak\end{minipage}}\end{center}}


\usepackage{amsfonts}
\usepackage{amssymb}
\usepackage{graphicx}
\usepackage{float}
\usepackage{booktabs}
\usepackage{parskip} % remove all the paragraph indents


\usepackage{setspace}
\usepackage[colorlinks=true]{hyperref}
\usepackage{textcomp} 
\usepackage{multicol} 

\usepackage{mathtools}          %loads amsmath as well added for the piece wise function
\DeclarePairedDelimiter\Floor\lfloor\rfloor
\DeclarePairedDelimiter\Ceil\lceil\rceil

 
\newcounter{NumberInTable}
\newcommand{\LTNUM}{\stepcounter{NumberInTable}{(\theNumberInTable)}}

\newcommand{\Laplace}[1]{\ensuremath{\mathcal{L}{\left[#1\right]}}}
\newcommand{\InvLap}[1]{\ensuremath{\mathcal{L}^{-1}{\left[#1\right]}}}
\renewcommand{\textuparrow}{$\uparrow$}

\begin{document}



	
\section{Two degree-of-freedom systems}

\subsection{Review of Matrix Algebra}

The \textbf{dot product} allows us to multiply matrices and is defined as:
\begin{eqnarray}
  \begin{bmatrix} a & b \\ c & d \end{bmatrix}\begin{bmatrix} e \\  f \end{bmatrix} = \begin{bmatrix} ae+bf \\ ce + df \end{bmatrix}
\end{eqnarray}
Another arrangement of the same principle, in a format more related to vibrations, is:
\begin{eqnarray}
  \begin{bmatrix} a_1+a_2 & b \\ c & d \end{bmatrix}\begin{bmatrix} e \\  f \end{bmatrix} = \begin{bmatrix} (a_1+a_2)e+bf \\ ce + df \end{bmatrix}
\end{eqnarray}
The \textbf{transpose of a matrix} is an  operator which flips a matrix over its diagonal. For a matrix $A$, the transpose $A^\text{T}$ can be written as:
\begin{eqnarray}
   A = \begin{bmatrix} a & b \\ c & d \\ e & f\end{bmatrix} \rightarrow A^\text{T} = \begin{bmatrix} a & c & e \\  b & d & f \end{bmatrix}
\end{eqnarray}
A matrix is \textbf{symmetric} if $A =A^\text{T}$. Therefore, symmetric matrix must be square and can be written as:
\begin{eqnarray}
   A = \begin{bmatrix} a & b &c \\ d & e & f\\ g & h & i \end{bmatrix} = A^\text{T} = \begin{bmatrix} a & d & g \\ b & e & h \\ c & f & i \end{bmatrix}\text{, where } b=d \text{, }c=g\text{, }f=h
\end{eqnarray}
The \textbf{determent} of a matrix is a value obtained from a square matrix and is often written as det($A$), det $A$, or $|A|$. For a 2 $\times$ 2 matrix this is defined as:
\begin{eqnarray}
\det (A) = ad-bc  \text{, when } A = \begin{bmatrix} a & b \\ c & d \end{bmatrix}
\end{eqnarray}
The inverse of a square matrix is such that $AA^{-1} = A^{-1}A=I$ where $I$ is the identity matrix:
\begin{eqnarray}
I = \begin{bmatrix} 1 & 0 \\ 0 & 1 \end{bmatrix} 
\end{eqnarray}
and the inverse of a 2 $\times$ 2 matrix is defined as:
\begin{eqnarray}
A^{-1} = \frac{1}{\det (A)} \begin{bmatrix} d & -b \\ -c & a \end{bmatrix} \text{, when } A = \begin{bmatrix} a & b \\ c & d \end{bmatrix}
\end{eqnarray}
Matrices that do not have an inverse are called a singular matrix. 



\subsection{Two Degree of Freedom Systems}

Consider the systems in the following figure:
\begin{figure}[H]
	\centering
	\includegraphics[width=0.5\textwidth]{../Figures/2_DOF_spring_mass_system_2.png}
\end{figure}
These are examples of 2 DOF systems because each system has two coordinate systems that express the movement of the mass. Another example of a 2-DOF system with two masses is show below: 
\begin{figure}[H]
	\centering
	\includegraphics[width=0.5\textwidth]{../Figures/2_DOF_spring_mass_system.png}
\end{figure}
where the two coordinates that describe the systems movements are $x_1$ and $x_2$. Before we start, let us consider the solution to the system shown. The solution consists of two equations, one for each mass. This solution will be derived in what follows and is expressed by the equations:
\begin{equation}
	x_1(t) = A_1 \sin (\omega_1 t + \phi_1 )u_{11}+ A_2 \sin (\omega_2 t + \phi_2 )u_{12} , \hspace{1cm} \omega_1 \text{ or } \omega_2 \neq 0
\end{equation}
\begin{equation}
	x_2(t) = A_1 \sin (\omega_1 t + \phi_1 )u_{21}+ A_2 \sin (\omega_2 t + \phi_2 )u_{22} , \hspace{1cm} \omega_1 \text{ or } \omega_2 \neq 0 \nonumber
\end{equation}
These two equations ban be written as a single equation in matrix form as:
\begin{equation}
	\mathbf{x}(t) = A_1 \sin (\omega_1 t + \phi_1 )\mathbf{u}_1 + A_2 \sin (\omega_2 t + \phi_2 )\mathbf{u}_2 , \hspace{1cm} \omega_1 \text{ or } \omega_2 \neq 0
\end{equation}
Where the bold text denotes vectors. Expanding these vectors shows, 
\begin{eqnarray}
 \mathbf{x}(t)=  \begin{bmatrix} x_1(t) \\  x_2(t) \end{bmatrix}, \hspace{2ex} \mathbf{u1}=  \begin{bmatrix} u_{11} \\  u_{21} \end{bmatrix}, \hspace{2ex} \mathbf{u_2}=  \begin{bmatrix} u_{12} \\  u_{22} \end{bmatrix}\text{, }
\end{eqnarray}
The four key components are:
\begin{enumerate}
\item $\omega_1$ and $\omega_2$ are the natural frequencies of the system.\textbf{ They are not the frequencies of the masses.} The formulation of the solution states that each mass in general oscillates at two frequencies. Furthermore, suppose that the initial conditions
are chosen such that A2 = 0.With such initial conditions each mass oscillates at only
one frequency, $\omega_1$.
\item $A_1$ and $A_2$ are the constants of integration and determine the amplitude of the system.
\item $\phi_1$ and $\phi_2$ represent the phase shift of the system
\item $\mathbf{u}_1$ and $\mathbf{u}_2$ are the first and second mode shapes of the system and couple the system together.
\end{enumerate}
To derive this solution for the system under consideration a FBD can be constructed for the forces acting on each mass. First we have to make the assumption that $x_1 < x_2$, this allows us to sat that $m_2$ pulls on $m_1$ and results in:
\begin{figure}[H]
	\centering
	\includegraphics[width=0.6\textwidth]{../Figures/2_DOF_spring_mass_system_FBD.png}
\end{figure}
Applying Newton's 2$^{nd}$ law and summing the forces on each mass in the horizontal direction yields:
\begin{eqnarray}
m_1\ddot{x}_1 &= & -k_1x_1 + k_2(x_2-x_1) \\
m_2\ddot{x}_2&= & -k_2(x_2-x_1)  \nonumber
\end{eqnarray}
These equations can be rearranged in terms of  $x_1$ and $x_2$ as:
\begin{eqnarray}
m_1\ddot{x}_1 +(k_1+k_2)x_1 -k_2x_2 =0 \\
m_2\ddot{x}_2 - k_2x_1 + k_2x_2 = 0 \nonumber
\end{eqnarray}
where these are two coupled second-order differential equations that each require two initial conditions to solve. These initial conditions can be obtained form the displacement and velocity terms as:
\begin{eqnarray}
x_1(0) = x_{10} \\
\dot{x}_1(0) = \dot{x}_{10} = v_{10} \nonumber \\ 
x_2(0) = x_{20} \nonumber \\ 
\dot{x}_2(0) = \dot{x}_{20} = v_{20} \nonumber
\end{eqnarray}
As before, these initial conditions will be the constants of integration used to solve the two second-order differential equations for the free response of each mass. There is a multitude of ways to solve these two coupled  second-order differential equations, however, here we will just consider a matrix notation solution. This matrix notation solution is used as this formulation is readably solved using computers and is expandable to more than 2 DOF.

To initiate the solution, let us first develop the matrix formulation of the two coupled ODEs.
\begin{eqnarray}
  \begin{bmatrix} m_1 & 0  \\  0 & m_2 \end{bmatrix}\begin{bmatrix} \ddot{x_1} \\  \ddot{x_2} \end{bmatrix} + \begin{bmatrix} k_1+k_2 & -k_2  \\  -k_2 & k_2 \end{bmatrix}\begin{bmatrix} x_1 \\  x_2 \end{bmatrix} = \begin{bmatrix} 0 \\  0 \end{bmatrix}
\end{eqnarray}
This equation can also be expressed as the vector equation 
\begin{equation}
M\mathbf{\ddot{x}} + K\mathbf{x} =0
\end{equation}
and is known as the EOM in vector form. In this formulation the mass matrix is defined as:
\begin{eqnarray}
 M=  \begin{bmatrix} m_1 & 0  \\  0 & m_2 \end{bmatrix}  
\end{eqnarray}
while the stiffness matrix is:
\begin{eqnarray}
 K=  \begin{bmatrix} k_1+k_2 & -k_2  \\  -k_2 & k_2 \end{bmatrix}
\end{eqnarray}
along with the displacement, velocity, and acceleration matrices:
\begin{eqnarray}
 \mathbf{x}=  \begin{bmatrix} x_1 \\  x_2 \end{bmatrix} , \hspace{2ex} \mathbf{\dot{x}}=  \begin{bmatrix} \dot{x}_1 \\  \dot{x}_2 \end{bmatrix}, \hspace{2ex} \mathbf{\ddot{x}}=  \begin{bmatrix} \ddot{x}_1 \\  \ddot{x}_2 \end{bmatrix}
\end{eqnarray}
Beyond these equations we can write the initial conditions as:
\begin{eqnarray}
\mathbf{x_0}=  \begin{bmatrix} x_1(0) \\  x_2(0) \end{bmatrix},  \hspace{1cm} \mathbf{\dot{x}_0}=  \begin{bmatrix} \dot{x}_1(0) \\  \dot{x}_2(0) \end{bmatrix}
\end{eqnarray}
This simple connection between vibration analysis and matrix analysis allows computers to be used to solve large and complicated vibration problems quickly.

Recall that the single-DOF version of the equation of motion was solved by assuming a harmonic solution and calculating the values of the constants in the assumed form. The same approach is used here to solve the equation of motion for the two-DOF system.  A solution is assumed in the form:

\begin{equation}
	\mathbf{x}(t) = \mathbf{u}e^{j\omega t}
\end{equation}
where $\mathbf{u}$ is a vector of constants to be demerited and can be written as:
\begin{eqnarray}
\mathbf{u}=  \begin{bmatrix} u_1 \\  u_2 \end{bmatrix}
\end{eqnarray}
From before, $\omega$ is also a constant to be determined. Again, $j=\sqrt{-1}$. In the same manner as before, $e^{j\omega t}$ represents harmonic motion as $e^{j\omega t} = \cos(\omega t) + j \sin(\omega t)$. Taking the derivatives of $\mathbf{x}(t) = \mathbf{u}e^{j\omega t}$ yields:
\begin{equation}
	\dot{\mathbf{x}}(t) = j\omega\mathbf{u}e^{j\omega t}
\end{equation}
\begin{equation}
	\ddot{\mathbf{x}}(t) = -\omega^2\mathbf{u}e^{j\omega t}
\end{equation}




Substituting this into the EOM in vector form ($M\mathbf{\ddot{x}} + K\mathbf{x} =0$) yields:
\begin{equation}
-\omega^2 M  \mathbf{u}e^{j\omega t} + K\mathbf{u}e^{j\omega t} =0
\end{equation}
or 
\begin{equation}
(-\omega^2 M  + K)\mathbf{u}e^{j\omega t} =0
\end{equation}
As $e^{j\omega t} \neq 0$ for any value of $t$ and not allowing $\mathbf{u}$ to be zero it can be demerited that $(-\omega^2 M  + K)$ must satisfy the vector equation. Therefore,
\begin{equation}
(-\omega^2 M  + K)\mathbf{u} =0, \hspace{1cm} \mathbf{u}\neq0
\end{equation}

for this homogeneous set of algebraic equations to have a nonzero solution for the vector $\mathbf{u}$, the inverse of the coefficient matrix $(-\omega^2 M  + K)$ must not exist. To see that this is the case, suppose that the inverse of $(-\omega^2 M  + K)$ does exist. Then multiplying both sides of the equation by $(-\omega^2 M  + K)^-1$ yields $\mathbf{u}=0$, a trivial solution, as it implies no motion. Hence the solution of equation depends in some way on the matrix inverse.

Applying the condition of singularity to the coefficient matrix of equation $(-\omega^2 M  + K)\mathbf{u} =0, \hspace{1ex} \mathbf{u}\neq0$ yields the result that for a nonzero solution of $\mathbf{u}$ to exist the following must be true:
\begin{equation}
\det(-\omega^2 M  + K) = 0
\end{equation}
which \textbf{yields one algebraic equation in one unknown ($\omega$)}. Substituting the values of the matrices $M$ and $K$ into this expression yields:
\begin{eqnarray}
\det\begin{bmatrix} -\omega^2 m_1 + k_1 + k_2 & -k_2  \\  -k_2 & -\omega^2 m_2 + k_2 \end{bmatrix}=0
\end{eqnarray}
Using the definition of the determinant yields that the unknown quantity $\omega^2$ must satisfy:
\begin{equation}
m_1 m_2 \omega^4 - (m_1 k_2 + m_2 k_1 + m_2 k_2)\omega^2 + k_1 k_2 = 0
\end{equation}
This expression is called the \textbf{characteristic equation} for the system and is used to determine the constants $\omega$ in the assumed form of the solution given by the assumed solution $\mathbf{x}(t) = \mathbf{u}e^{j\omega t}$ once the values of the physical parameters $m_1$, $m_2$, $k_1$, and $k_2$ are known. Note that $\omega$ is not in the characteristic equation, therefore, solving for $\omega$  will be done by factoring the equation above to obtain two solutions $\omega_1$ and $\omega_2$. The characteristic equation is in the form of the quadratic formula if you set $x=\omega^2$, as
\begin{equation}
ax^2 + bx +c = 0
\end{equation}

After finding the value of $\omega$ using the characteristic equation, the values in $\mathbf{u}$ can be found using equation $(-\omega^2 M  + K)\mathbf{u} =0, \hspace{1ex} \mathbf{u}\neq0$ for each value of $\omega^2$. That is, for both $\omega_1$ and $\omega_2$ there is a a vector  $\mathbf{u}$ that satisfies the equation. These solutions can be written as:
\begin{equation}
	(-\omega_1^2 M  + K)\mathbf{u}_1 =0
\end{equation}
and 
\begin{equation}
	(-\omega_2^2 M  + K)\mathbf{u}_2 =0
\end{equation}
These expressions can be solved for the \textbf{direction of the vectors $\mathbf{u}_1$ and $\mathbf{u}_2$, but not for the magnitude.} To see that this is true, note that if $\mathbf{u}_1$, satisfies equation, so does the vector $a\mathbf{u}_1$ where $a$ is any nonzero number. Hence the vectors satisfying the above are of arbitrary magnitude.
 
The values obtained for $\mathbf{u}_1$ and $\mathbf{u}_2$ can now be combined with the assumed solution
\begin{equation}
	\mathbf{x}(t) = \mathbf{u}e^{j\omega t}
\end{equation}
to form a set of solution,
\begin{equation}
	\mathbf{x}(t) = \mathbf{u}_1e^{-j\omega_1 t}, \hspace{2ex} \mathbf{u}_1e^{j\omega_1 t}, \hspace{2ex} \mathbf{u}_2e^{-j\omega_2 t}, \hspace{2ex} \mathbf{u}_2e^{j\omega_2 t}
\end{equation}
Since the equation to be solved is linear, the solution is the sum of these solutions. This results in:
\begin{equation}
	\mathbf{x}(t) = (a e^{j\omega_1 t} + b e^{-j\omega_1 t})\mathbf{u}_1 +(c e^{j\omega_2 t} + d e^{-j\omega_2 t})\mathbf{u}_2
\end{equation}
where $a$, $b$, $c$, and $d$ are the arbitrary constants of integration to be determined by the initial conditions. Applying Euler's formulas for the sin functions (where $\omega_1 \text{ or } \omega_2 \neq 0$) reorganizes this equation as:
\begin{equation}
	\mathbf{x}(t) = A_1 \sin (\omega_1 t + \phi_1 )\mathbf{u}_1 + A_2 \sin (\omega_2 t + \phi_2 )\mathbf{u}_2 , \hspace{1cm} \omega_1 \text{ or } \omega_2 \neq 0
\end{equation}
Another way to write this equation is in the form
\begin{equation}
	 \begin{bmatrix} x_1(t) \\  x_2(t) \end{bmatrix} =  \begin{bmatrix} \mathbf{u}_1 & \mathbf{u}_2 \end{bmatrix}
	 \begin{bmatrix} A_1 \sin (\omega_1 t + \phi_1 )\\ A_2 \sin (\omega_2 t + \phi_2 )\end{bmatrix}, \hspace{1cm} \omega_1 \text{ or } \omega_2 \neq 0
\end{equation}

Where the values for $A_1$ and $A_2$ can be obtained by setting applying the boundary conditions as and taking the derivatives of the equations as done in the 1-DOF problems. The forms of final equations give physical meaning to the solution. It states that each mass oscillates at the natural frequencies of the system $\omega_1$ and $\omega_2$. Furthermore, suppose that the initial conditions are chosen such that $A_2$ = 0. With such initial conditions each mass oscillates at only one frequency,  $\omega_1$, and the relative positions of the masses at any given instant of
time are determined by the elements of the vector $\mathbf{u}_1$. Hence $\mathbf{u}_1$, is called the \textbf{first mode shape of the system.} Similarly, if the initial conditions are chosen such that $A_1$ is zero, both coordinates oscillate at frequency $\omega_2$, with relative positions given by the vector $\mathbf{u}_2$, called the \textbf{second mode shape}. The concepts of natural frequencies and mode shapes are extremely important and form one of the major ideas used in vibration studies.


\begin{example}

Considering the following system:
\begin{figure}[H]
	\centering
	\includegraphics[width=0.5\textwidth]{../Figures/2_DOF_spring_mass_system.png}
\end{figure}
Calculate response for the system if $m_1$=9 kg, $m_2$=1 kg, $k_1$ = 24 N/m, and $k_2$ = 3 N/m with the initial conditions $x_{10}=1$ mm, $v_{10}=0$ mm/s, $x_{20}=0$ mm, and $v_{20}=0$ mm/s.

\textbf{Solution}

We have already obtained a characteristic equation for this system, given as:
\begin{equation}
m_1 m_2 \omega^4 - (m_1 k_2 + m_2 k_1 + m_2 k_2)\omega^2 + k_1 k_2 = 0
\end{equation}
Substituting our values into this obtains:
\begin{equation}
9 \cdot 1 \omega^4 - (9 \cdot 3 + 1 \cdot 24 + 1 \cdot 3)\omega^2 + 24 \cdot 3 = 0
\end{equation}
or
\begin{equation}
\omega^4 - 6\omega^2 + 8 =0
\end{equation}
This can then be factored into:
\begin{equation}
(\omega^2-2)(\omega^2-4)=0
\end{equation}
This results in solutions of $\omega^2_1 = 2$ and $\omega^2_2 = 4$. Leading to:
\begin{equation}
\omega_1 \pm \sqrt{2} \text{ rad/sec}, \hspace{2ex} \omega_2 \pm 2 \text{ rad/sec}
\end{equation}
Next, we need to obtain solutions for $\mathbf{u}_1$ and $\mathbf{u}_2$. Having solved for $\omega_1$ and $\omega_2$ we can obtained. First, knowing $\mathbf{u}_1 = [u_{11} u_{21}]^\text{T}$ and using $\omega_1 = \sqrt{2}$ and the follwoing equation:
\begin{equation}
	(-\omega_1^2 M  + K)\mathbf{u}_1 =0
\end{equation}
yields
simplified to
\begin{equation}
	 \bigg(-2\begin{bmatrix} 9 & 0 \\   0  & 1 \end{bmatrix} + \begin{bmatrix} 24+3 & -3 \\    -3  & 3 \end{bmatrix}\bigg)\begin{bmatrix} u_{11}\\ u_{21}\end{bmatrix} = \begin{bmatrix} 0\\ 0\end{bmatrix}
\end{equation}
simplified to
\begin{equation}
	 \begin{bmatrix} 27-9\cdot 2 & -3 \\    -3  & 3-2 \end{bmatrix} 
	 \begin{bmatrix} u_{11}\\ u_{21}\end{bmatrix}=\begin{bmatrix} 0\\ 0\end{bmatrix}
\end{equation}
or
\begin{equation}
	 \begin{bmatrix} 9 & -3 \\    -3  & 1 \end{bmatrix} 
	 \begin{bmatrix} u_{11}\\ u_{21}\end{bmatrix}=\begin{bmatrix} 0\\ 0\end{bmatrix}
\end{equation}
Taking the dot product of the matrix equation yields:
\begin{equation}
	9u_{11} -3u_{21}=0 \text{, and } -3u_{11} + u_{21}=0
\end{equation}
Both of these equations yield the same equation, that is:
\begin{equation}
	\frac{u_{11}}{u_{21}} =\frac{1}{3}
\end{equation}
As mentioned before, only the ratio of the elements is determined here. To show this is true it is easily seen that:
\begin{equation}
	u_{11}=u_{21}\frac{1}{3} \rightarrow  a u_{11}= a u_{21}\frac{1}{3} 
\end{equation}
To obtain a numerical value, we arbitrarily assign a value to one of the elements. Here, let $u_{21}=1$ so  let $u_{21}=1/3$. Therefore, 
\begin{equation}
	 \textbf{u}_1 = \begin{bmatrix} \frac{1}{3}\\ 1\end{bmatrix}
\end{equation}
The same processes can be used for obtaining $\textbf{u}_1$ using $\omega_2=2$, this results in:
\begin{equation}
	 \begin{bmatrix} -9 & -3 \\    -3  & -1 \end{bmatrix} 
	 \begin{bmatrix} u_{12}\\ u_{22}\end{bmatrix}=\begin{bmatrix} 0\\ 0\end{bmatrix}
\end{equation}
Taking the dot product of the matrix equation yields:
\begin{equation}
	-9u_{12} -3u_{22}=0 \text{, and } -3u_{12} - u_{22}=0
\end{equation}
Both of these equations yield the same equation, that is:
\begin{equation}
	\frac{u_{12}}{u_{22}} =-\frac{1}{3}
\end{equation}
Again, assuming $u_{22}=1$  this can be rearranged into $\textbf{u}_2$ as:
\begin{equation}
	 \textbf{u}_2 = \begin{bmatrix} -\frac{1}{3}\\ 1\end{bmatrix}
\end{equation}
Where $\mathbf{u}_1$ and $\mathbf{u}_2$ represent only the directions ans shape of the mode shapes and not the magnitude of the mode shapes. 
Now that we have the mode shapes, we can solve for the initial conditions $A_1$ and $A_2$. To do this, let us use the following formulation of the solution:
\begin{equation}
	 \begin{bmatrix} x_1(t) \\  x_2(t) \end{bmatrix} =  \begin{bmatrix} \mathbf{u}_1 & \mathbf{u}_2 \end{bmatrix}
	 \begin{bmatrix} A_1 \sin (\omega_1 t + \phi_1 )\\ A_2 \sin (\omega_2 t + \phi_2 )\end{bmatrix}, \hspace{1cm} \omega_1 \text{ or } \omega_2 \neq 0
\end{equation}
Adding our values for the problem at $t=0$ this becomes:
\begin{equation}
	 \begin{bmatrix} 1 \\  0 \end{bmatrix} =  \begin{bmatrix} \frac{1}{3} & -\frac{1}{3} \\ 1 & 1 \end{bmatrix}
	 \begin{bmatrix} A_1 \sin (\sqrt{2} t + \phi_1 )\\ A_2 \sin (2 t + \phi_2 )\end{bmatrix}
\end{equation}
and after applying the dot product:
\begin{equation}
	 \begin{bmatrix} 1 \\  0 \end{bmatrix} =  \begin{bmatrix} \frac{1}{3}A_1 \sin (\phi_1 ) -\frac{1}{3}A_2 \sin (\phi_2)\\ A_1 \sin (\phi_1 )+A_2 \sin (\phi_2 )\end{bmatrix}
\end{equation}
Next we can differentiate the equation for $x(t)$ to obtain the velocity solution. Adding our values for the problem at $t=0$ obtains:
\begin{equation}
	 \begin{bmatrix} \dot{x}_1(0) \\  \dot{x}_2(0) \end{bmatrix}  = \begin{bmatrix} v_{10} \\  v_{20} \end{bmatrix} = \begin{bmatrix} 0 \\  0 \end{bmatrix} =   \begin{bmatrix} \frac{\sqrt{2}}{3}A_1 \cos (\phi_1 ) -\frac{2}{3}A_2 \cos (\phi_2)\\ \sqrt{2}A_1 \cos (\phi_1 )+2 A_2 \cos (\phi_2 )\end{bmatrix}
\end{equation}
Now that we have 4 equations for 4 unknowns we can use these equations to solve for $A_1$,  $A_2$, $\phi_1$,  and $\phi_2$. The 4 equations are:
\begin{equation}
3= A_1 \sin (\phi_1 ) - A_2 \sin (\phi_2)
\end{equation}
\begin{equation}
0= A_1 \sin (\phi_1 ) + A_2 \sin (\phi_2)
\end{equation}
\begin{equation}
0= \sqrt{2}A_1 \cos (\phi_1 ) - 2A_2 \cos (\phi_2)
\end{equation}
\begin{equation}
0= \sqrt{2}A_1 \cos (\phi_1 ) + 2A_2 \cos (\phi_2)
\end{equation}
Setting these last two equations equal to each other yields:
\begin{equation}
0= \sqrt{2}A_1 \cos (\phi_1 ) + 2A_2 \cos (\phi_2) = \sqrt{2}A_1 \cos (\phi_1 ) - 2A_2 \cos (\phi_2)
\end{equation}
or:
\begin{equation}
0= - 4A_2 \cos (\phi_2)
\end{equation}
For this equation to be true, $\phi_2=\frac{\pi}{2}$. Therefore, applying this to $0= \sqrt{2}A_1 \cos (\phi_1 ) + 2A_2 \cos (\phi_2)$ results in:
\begin{equation}
0= \sqrt{2}A_1 \cos (\phi_1 )
\end{equation}
where again, for this equation to be true, $\phi_1=\frac{\pi}{2}$. Now the first two equations become:
\begin{equation}
3= A_1 - A_2 
\end{equation}
\begin{equation}
0= A_1 + A_2 
\end{equation}
Where this shows us that $A_1 = \frac{3}{2}$ mm and $A_2 = -\frac{3}{2}$. Therefore, now that we have the initial conditions we can find a solution for the temporal response of each mass. Using the equations from before:
\begin{equation}
	x_1(t) = A_1 \sin (\omega_1 t + \phi_1 )u_{11} + A_2 \sin (\omega_2 t + \phi_2 )u_{12}
\end{equation}
\begin{equation}
	x_2(t) = A_1 \sin (\omega_1 t + \phi_1 )u_{21} + A_2 \sin (\omega_2 t + \phi_2 )u_{22}
\end{equation}
And applying our obtained values
\begin{equation}
	x_1(t) = \frac{3}{2} \sin (\sqrt{2} t + \frac{\pi}{2} )\frac{1}{3} + \bigg(-\frac{3}{2}\bigg) \sin (2 t + \frac{\pi}{2} ) \bigg(-\frac{1}{3}\bigg)
\end{equation}
\begin{equation}
	x_2(t) = \frac{3}{2} \sin (\sqrt{2} t + \frac{\pi}{2} ) + \bigg(-\frac{3}{2}\bigg) \sin (2 t + \frac{\pi}{2} )
\end{equation}
results in:
\begin{equation}
	x_1(t) = \frac{1}{2} \bigg(  \sin (\sqrt{2} t + \frac{\pi}{2} ) + \sin (2 t + \frac{\pi}{2} ) \bigg)
\end{equation}
\begin{equation}
	x_2(t) = \frac{3}{2}  \bigg( \sin (\sqrt{2} t + \frac{\pi}{2} ) -\sin (2 t + \frac{\pi}{2} ) \bigg)
\end{equation}
These results can be plotted as:
\begin{figure}[H]
	\centering
	\includegraphics[width=0.9\textwidth]{../Figures/Example_1_2_DOF_response.png}
\end{figure}

\end{example}

\begin{example}


Let's take a look at the same system, but this time we will use the second mass to ``tune'' the general natural frequency of the fist mass. To do so, use 
Considering the following system:
\begin{figure}[H]
	\centering
	\includegraphics[width=0.5\textwidth]{../Figures/2_DOF_spring_mass_system.png}
\end{figure}
To do so, use  $m_1$=10 kg, $m_2$=5 kg, $k_1$ = 10 N/m, and $k_2$ = 25 N/m with the initial conditions $x_{10}=1$ mm, $v_{10}=0$ mm/s, $x_{20}=0$ mm, and $v_{20}=0$ mm/s.

\textbf{Solution}

We have already obtained a characteristic equation for this system, given as:
\begin{equation}
m_1 m_2 \omega^4 - (m_1 k_2 + m_2 k_1 + m_2 k_2)\omega^2 + k_1 k_2 = 0
\end{equation}
This results in solutions of $\omega^2_1 = 2$ and $\omega^2_2 = 4$. Leading to:
\begin{equation}
\omega_1 \pm 0.797 \text{ rad/sec}, \hspace{2ex} \omega_2 \pm 2.08 \text{ rad/sec}
\end{equation}
Next, we need to obtain solutions for $\mathbf{u}_1$ and $\mathbf{u}_2$. Here, let $u_{21}=1$ so, 
\begin{equation}
	 \textbf{u}_1 = \begin{bmatrix} 0.87\\ 1\end{bmatrix}
\end{equation}
The same processes can be used for obtaining $\textbf{u}_1$. Again, assuming $u_{22}=1$ this can be rearranged into $\textbf{u}_2$ as:
\begin{equation}
	 \textbf{u}_2 = \begin{bmatrix} -0.57 \\ 1\end{bmatrix}
\end{equation}
Now that we have the mode shapes, we can solve for the initial conditions $A_1$ and $A_2$. Resulting in $A_1 = 0.692$ mm and $A_2 = -0.692$. Therefore, now that we have the initial conditions we can find a solution for the temporal response of each mass. Using the equations from before:
\begin{equation}
	x_1(t) = A_1 \sin (\omega_1 t + \phi_1 )u_{11} + A_2 \sin (\omega_2 t + \phi_2 )u_{12}
\end{equation}
\begin{equation}
	x_2(t) = A_1 \sin (\omega_1 t + \phi_1 )u_{21} + A_2 \sin (\omega_2 t + \phi_2 )u_{22}
\end{equation}
These results can be plotted as:
\begin{figure}[H]
	\centering
	\includegraphics[width=0.9\textwidth]{../Figures/Example_2_2_DOF_response.png}
\end{figure}

It can be seen here that the ``general'' natural frequency the main mass has be reduced, all without changing the mass of the stiffness $k_1$. This was done by passing energy from the main mass $m_1$ to the much smaller mass $m_2$. Such an example can be useful when we want to alter the vibrations of a system but we are unable to alter the system itself. If damping were to be added to the system, therefore allowing energy to be extracted from the system, even more impressive results could be obtained. 


\end{example}


\end{document}














